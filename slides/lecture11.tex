%%%%%%%%%%%%%%%%%%%%%%%%%%%%%%%%%%%%%%%%
\section{Introduction: Probability as Area}
%%%%%%%%%%%%%%%%%%%%%%%%%%%%%%%%%%%%%%%%

\begin{frame}
\frametitle{Continuous RVs -- What Changes?}
	\begin{enumerate}
\item Probability Density Functions replace Probability Mass Functions 
\item Integrals Replace Sums
\end{enumerate}
\begin{alertblock}{Everything Else is Essentially Unchanged!}\end{alertblock}


\end{frame}

%%%%%%%%%%%%%%%%%%%%%%%%%%%%%%%%%%%%%%%%


\begin{frame}
\frametitle{What is the probability of ``Yellow?''\hfill \includegraphics[scale = 0.05]{./images/clicker}}
\centering
	\includegraphics[scale = 0.6]{./images/twister}

\end{frame}
%%%%%%%%%%%%%%%%%%%%%%%%%%%%%%%%%%%%%%%%

%
%\begin{frame}
%\frametitle{What is the probability of ``Right Hand Blue?'' \hfill \includegraphics[scale = 0.05]{./images/clicker}}
%\centering
%	\includegraphics[scale = 0.6]{./images/twister}
%
%\end{frame}
%%%%%%%%%%%%%%%%%%%%%%%%%%%%%%%%%%%%%%%%
%\begin{frame}
%\frametitle{What is the probability that the spinner lands in any \emph{particular} place?\hfill \includegraphics[scale = 0.05]{./images/clicker}}
%
%\begin{center}
%	\includegraphics[scale = 0.6]{./images/twister}
%\end{center}
%\end{frame}




%%%%%%%%%%%%%%%%%%%%%%%%%%%%%%%%%%%%%%%%

\begin{frame}
\frametitle{From Twister to Density -- Probability as \emph{Area}}

\centering
	\includegraphics[scale = 0.5]{./images/twister_density}


\alert{For continuous RVs, probability is defined as \emph{area under a curve}. Zero area means zero probability!}


\end{frame}


%%%%%%%%%%%%%%%%%%%%%%%%%%%%%%%%%%%%%%%%

\section{Probability Density Function (PDF)}
%%%%%%%%%%%%%%%%%%%%%%%%%%%%%%%%%%%%%%%%
\begin{frame}
\frametitle{Probability Density Function (PDF)}
For a continuous random variable $X$, 
	$$P(a \leq X \leq b) = \int_a^b f(x) \; dx$$
where $f(x)$ is the \emph{probability density function} for $X$. 
\vspace{2em}

\begin{alertblock}{Extremely Important}
For any realization $x$, $P(X=x) = 0$ since $\int_a^a f(x) dx = 0$.
In other words, zero area means zero probability!
\end{alertblock}
\end{frame}
%%%%%%%%%%%%%%%%%%%%%%%%%%%%%%%%%%%%%%%%
\begin{frame}
  \frametitle{For a Continuous RV, Zero Probability $\neq$ Impossible}

    It is \emph{crucial} to specify the support set of a continuous RV:
  \begin{itemize}
    \item Any $x$ outside the support set of $X$ is \emph{impossible}.
    \item Any $x$ in the support set of $X$ is a \emph{possible outcome} even though $P(X=x) = 0$ for all $x$.
  \end{itemize}

  \vspace{1em}

  \alert{There is no way around this slightly awkward situation: it is a consequence of defining probability as the \emph{area under a curve}.}
  
\end{frame}
%%%%%%%%%%%%%%%%%%%%%%%%%%%%%%%%%%%%%%%%
\begin{frame}
  \frametitle{Properties of PDFs}

\begin{enumerate}
  \item $f(x)\geq 0$ for all $x$ in the support of $X$ and zero otherwise.
\item $\int_{-\infty}^\infty f(x) \; dx = 1$ 
\end{enumerate}

\pause

\vspace{1em}

\begin{alertblock}{Warning: $f(x)$ is \emph{not a probability}}
  Can have $f(x)>1$ for some $x$ as long as $\int_{-\infty}^\infty f(x) dx =1$.
\end{alertblock}

\pause

\vspace{1em}

\begin{block}{Relating the CDF to the PDF}
  \vspace{-1em}
  \[
  F(x_0) \equiv P(X\leq x_0) = \int_{-\infty}^{x_0} f(x) \; dx 
\]
\end{block}

\end{frame}
%%%%%%%%%%%%%%%%%%%%%%%%%%%%%%%%%%%%%%%%
\begin{frame}[fragile]
  \frametitle{Example: Suppose $X$ has Support Set $[0,1]$}

  Let $f(x) = 6x(1-x)$ for $x \in [0,1]$ and zero otherwise.

  \footnotesize
\begin{knitrout}
\definecolor{shadecolor}{rgb}{0.969, 0.969, 0.969}\color{fgcolor}\begin{kframe}
\begin{alltt}
\hlkwd{curve}\hlstd{(}\hlnum{6} \hlopt{*} \hlstd{x} \hlopt{*} \hlstd{(}\hlnum{1} \hlopt{-} \hlstd{x),} \hlkwc{from} \hlstd{=} \hlnum{0}\hlstd{,} \hlkwc{to} \hlstd{=} \hlnum{1}\hlstd{,} \hlkwc{ylab} \hlstd{=} \hlstr{'f(x)'}\hlstd{)}
\hlkwd{abline}\hlstd{(}\hlkwc{h} \hlstd{=} \hlnum{1}\hlstd{,} \hlkwc{lty} \hlstd{=} \hlnum{2}\hlstd{)}
\end{alltt}
\end{kframe}
\includegraphics[width=\maxwidth]{figure/pdf-1} 

\end{knitrout}


\end{frame}
%%%%%%%%%%%%%%%%%%%%%%%%%%%%%%%%%%%%%%%%
\begin{frame}
  \frametitle{Example: Suppose $X$ has Support Set $[0,1]$}

  Let $f(x) = 6x(1-x)$ for $x \in [0,1]$ and zero otherwise.

\vspace{2em}

\begin{alertblock}{Is $f$ a valid PDF?}
  \begin{enumerate}
    \item Is $f(x) \geq 0$ for $x \in [0,1]$ and zero otherwise? 
    \item Does the total area under $f$ equal one?
  \end{enumerate}
  \begin{eqnarray*}
    \int_{-\infty}^{\infty} f(x)dx &= \pause \displaystyle \int_{0}^1 6x(1-x)dx = \pause 6\int_0^1 (x - x^2)dx\\
    &= \pause\displaystyle 6\left.\left( \frac{x^2}{2} - \frac{x^3}{3} \right)\right|_0^1 \pause = 1
  \end{eqnarray*}
\end{alertblock}

\hfill\alert{So yes, $f$ is a valid PDF \checkmark}


%\begin{block}{We say that $X \sim \text{Beta}(\alpha, \beta)$ if}
%  \begin{enumerate}[(i)]
%    \item $X$ has support support set $[0,1]$, and 
%    \item $X$ has PDF $f(x) = C x^\alpha (1 - x)^\beta$
%  \end{enumerate}
%  where $(\alpha, \beta)$ are parameters that control the shape of the pdf and $C$ is the appropriate constant such that $\int_{0}^1 f(x) dx = 1$.
%\end{block}

\end{frame}
%%%%%%%%%%%%%%%%%%%%%%%%%%%%%%%%%%%%%%%%
\begin{frame}[fragile]
  \frametitle{Integrating a Function in R}
\begin{knitrout}
\definecolor{shadecolor}{rgb}{0.969, 0.969, 0.969}\color{fgcolor}\begin{kframe}
\begin{alltt}
\hlstd{pdf} \hlkwb{<-} \hlkwa{function}\hlstd{(}\hlkwc{x}\hlstd{) \{}
  \hlnum{6} \hlopt{*} \hlstd{x} \hlopt{*} \hlstd{(}\hlnum{1} \hlopt{-} \hlstd{x)}
\hlstd{\}}

\hlkwd{integrate}\hlstd{(pdf,} \hlkwc{lower} \hlstd{=} \hlnum{0}\hlstd{,} \hlkwc{upper} \hlstd{=} \hlnum{1}\hlstd{)}
\end{alltt}
\begin{verbatim}
## 1 with absolute error < 1.1e-14
\end{verbatim}
\end{kframe}
\end{knitrout}

\hfill\alert{You can use this to check your work!}

\end{frame}
%%%%%%%%%%%%%%%%%%%%%%%%%%%%%%%%%%%%%%%%
\begin{frame}
  \frametitle{Example: $f(x) = 6x(1-x)$ for $x \in [0,1]$, zero otherwise.}

  \begin{block}{What is the CDF of $X$?}

    \begin{eqnarray*}
      F(x_0) &\equiv& P(X \leq x_0) = \int_{-\infty}^{x_0} f(x)\; dx = \int_{0}^{x_0} 6x(1-x) \; dx\\  \pause
      &=& \left. 6\left( \frac{x^2}{2} - \frac{x^3}{3} \right)\right|_0^{x_0} = 3x_0^2 - 2x_0^3 
    \end{eqnarray*}

    \pause
  
  \[
    F(x_0) = \left\{
    \begin{array}{lr}
      0, & x_0 < 0\\
      3x_0^2 - 2x_0^3, & 0 \leq x_0 \leq 1\\
      1, & x_0 > 1
    \end{array}
    \right.
  \]

\end{block}



\end{frame}
%%%%%%%%%%%%%%%%%%%%%%%%%%%%%%%%%%%%%%%%
\begin{frame}[fragile]
\begin{knitrout}
\definecolor{shadecolor}{rgb}{0.969, 0.969, 0.969}\color{fgcolor}\begin{kframe}
\begin{alltt}
\hlkwd{par}\hlstd{(}\hlkwc{mfrow} \hlstd{=} \hlkwd{c}\hlstd{(}\hlnum{1}\hlstd{,}\hlnum{2}\hlstd{))}
\hlkwd{curve}\hlstd{(}\hlnum{6} \hlopt{*} \hlstd{x} \hlopt{*} \hlstd{(}\hlnum{1} \hlopt{-} \hlstd{x),} \hlkwc{from} \hlstd{=} \hlnum{0}\hlstd{,} \hlkwc{to} \hlstd{=} \hlnum{1}\hlstd{,} \hlkwc{ylab} \hlstd{=} \hlstr{'f(x)'}\hlstd{)}
\hlkwd{curve}\hlstd{(}\hlnum{3} \hlopt{*} \hlstd{x}\hlopt{^}\hlnum{2} \hlopt{-} \hlnum{2} \hlopt{*} \hlstd{x}\hlopt{^}\hlnum{3}\hlstd{,} \hlkwc{from} \hlstd{=} \hlnum{0}\hlstd{,} \hlkwc{to} \hlstd{=} \hlnum{1}\hlstd{,} \hlkwc{ylab} \hlstd{=} \hlstr{'F(x)'}\hlstd{)}
\end{alltt}
\end{kframe}
\includegraphics[width=\maxwidth]{figure/pdf_cdf-1} 
\begin{kframe}\begin{alltt}
\hlkwd{par}\hlstd{(}\hlkwc{mfrow} \hlstd{=} \hlkwd{c}\hlstd{(}\hlnum{1}\hlstd{,}\hlnum{1}\hlstd{))}
\end{alltt}
\end{kframe}
\end{knitrout}

\end{frame}
%%%%%%%%%%%%%%%%%%%%%%%%%%%%%%%%%%%%%%%%
\section{Relating the PDF to the CDF}
%%%%%%%%%%%%%%%%%%%%%%%%%%%%%%%%%%%%%%%%


\begin{frame}
\frametitle{Relationship between PDF and CDF}

\begin{block}{Integrate PDF to get CDF}
	\[
    F(x_0) = P(X\leq x_0) = \int_{-\infty}^{x_0} f(x)\; dx
  \]
\end{block}

\begin{block}{Differentiate CDF to get PDF}
 	\[
    f(x) =\frac{d}{dx}F(x)
  \]
\end{block}

\alert{This is just the First Fundamental Theorem of Calculus.}
\end{frame}


%%%%%%%%%%%%%%%%%%%%%%%%%%%%%%%%%%%%%%%%
\begin{frame}
  \frametitle{Example: $f(x) = 6x(1-x)$ for $x \in [0,1]$, zero otherwise.}

  \begin{block}{Differentiate CDF to get PDF}
    \begin{eqnarray*}
      f(x) &=& \frac{d}{dx}F(x) = \frac{d}{dx} \left( 3x^2 - 2x^3 \right)\\
      &=& \pause 6x - 6x^2\\
      &=& \pause 6x(1 - x)
    \end{eqnarray*}
  \end{block}
\end{frame}
%%%%%%%%%%%%%%%%%%%%%%%%%%%%%%%%%%%%%%%%
\section{Calculating the Probability of an Interval}
%%%%%%%%%%%%%%%%%%%%%%%%%%%%%%%%%%%%%%%%
\begin{frame}
\frametitle{Key Idea: Probability of an Interval for a Continuous RV}

$$\boxed{P(a\leq X \leq b) = \int_a^b f(x) \; dx = F(b) - F(a)}$$

\vspace{2em}
\alert{This is just the Second Fundamental Theorem of Calculus.}
\end{frame}
%%%%%%%%%%%%%%%%%%%%%%%%%%%%%%%%%%%%%%%%
\begin{frame}[fragile]
  \frametitle{Example: $f(x) = 6x(1-x)$ for $x \in [0,1]$, zero otherwise.}

  Two equivalent ways of calculating $P(0.2 \leq X \leq 0.6)$

\begin{knitrout}
\definecolor{shadecolor}{rgb}{0.969, 0.969, 0.969}\color{fgcolor}\begin{kframe}
\begin{alltt}
\hlstd{cdf} \hlkwb{<-} \hlkwa{function}\hlstd{(}\hlkwc{x0}\hlstd{) \{}
  \hlnum{3} \hlopt{*} \hlstd{x0}\hlopt{^}\hlnum{2} \hlopt{-} \hlnum{2} \hlopt{*} \hlstd{x0}\hlopt{^}\hlnum{3}
\hlstd{\}}
\hlkwd{cdf}\hlstd{(}\hlnum{0.6}\hlstd{)} \hlopt{-} \hlkwd{cdf}\hlstd{(}\hlnum{0.2}\hlstd{)}
\end{alltt}
\begin{verbatim}
## [1] 0.544
\end{verbatim}
\begin{alltt}
\hlkwd{integrate}\hlstd{(pdf,} \hlkwc{lower} \hlstd{=} \hlnum{0.2}\hlstd{,} \hlkwc{upper} \hlstd{=} \hlnum{0.6}\hlstd{)}
\end{alltt}
\begin{verbatim}
## 0.544 with absolute error < 6e-15
\end{verbatim}
\end{kframe}
\end{knitrout}
  %\begin{block}{Calculating $P(0.1\leq X \leq 0.4)$ using $F(x_0) = 3x_0^2 - 2x_0^3$}
  %  \begin{align*}
  %    P(0.1 \leq X \leq 0.4) &= F(0.4) - F(0.1) \\
  %    &= \left[ 3\times (0.4)^2 - 2 \times (0.4)^3 \right] - \left[ 3 \times (0.1)^2 - (0.1)^3 \right]\\
  %    &= ???
  %  \end{align*}
  %\end{block}

\end{frame}
%%%%%%%%%%%%%%%%%%%%%%%%%%%%%%%%%%%%%%%%
\begin{frame}[fragile]
  \frametitle{Example: $f(x) = 6x(1-x)$ for $x \in [0,1]$, zero otherwise.}
\begin{knitrout}
\definecolor{shadecolor}{rgb}{0.969, 0.969, 0.969}\color{fgcolor}
\includegraphics[width=\maxwidth]{figure/unnamed-chunk-1-1} 

\end{knitrout}

\begin{center}
  \alert{$P(0.2 \leq X \leq 0.6) = 0.544$}
\end{center}
\end{frame}
%%%%%%%%%%%%%%%%%%%%%%%%%%%%%%%%%%%%%%%%
\section{Calculating Expected Value for Continuous RVs}
%%%%%%%%%%%%%%%%%%%%%%%%%%%%%%%%%%%%%%%%

\begin{frame}
\frametitle{Expected Value for Continuous RVs}
\[\boxed{E[X] = \int_{-\infty}^\infty x f(x) \; dx}\]

\[\boxed{E[g(X)] = \int_{-\infty}^\infty g(x) f(x) \; dx}\]

\vspace{2em}
\hfill\alert{Integrals Replace Sums!}
\end{frame}

%%%%%%%%%%%%%%%%%%%%%%%%%%%%%%%%%%%%%%%%
\begin{frame}
\frametitle{What about all those rules for expected value?}
\begin{itemize}
  \item The only difference between expectation for continuous versus discrete is how we do the \emph{calculation}.
  \item Sum for discrete; integral for continuous.
  \item All \emph{properties} of expected value \alert{continue to hold!}
  \item Includes linearity, shortcut for variance, etc.
\end{itemize}
\end{frame}

%%%%%%%%%%%%%%%%%%%%%%%%%%%%%%%%%%%%%%%%
\begin{frame}
\frametitle{Variance of Continuous RV}

$$\boxed{Var(X) = \int_{-\infty}^{\infty} (x - \mu)^2 f(x) \; dx}$$

\vspace{2em}
where
$$\mu = E[X]=\int_{-\infty}^\infty x f(x) \; dx $$

\vspace{2em}
\alert{Shortcut formula still holds for continuous RVs!}
	$$Var(X) = E[X^2] - \left(E[X]\right)^2$$
\end{frame}
%%%%%%%%%%%%%%%%%%%%%%%%%%%%%%%%%%%%%%%%
\begin{frame}
  \frametitle{Example: $f(x) = 6x(1-x)$ for $x \in [0,1]$, zero otherwise.}

  \small
    \begin{eqnarray*}
      E[X] &=& \int_{-\infty}^{\infty} x f(x)\; dx =\pause \int_0^1 x \cdot 6x(1-x) \; dx=\pause \left. 6\left( \frac{x^3}{3} - \frac{x^4}{4} \right)\right|_{0}^1 = \frac{1}{2}\\ \\ 
      \pause
      E[X^2] &=& \int_{-\infty}^{\infty} x^2 f(x)\; dx =\pause \int_0^1 x^2 \cdot 6x(1-x) = \pause \left. 6\left( \frac{x^4}{4}\; dx - \frac{x^5}{5} \right) \right|_0^1 = \frac{3}{10} \\ \\ 
      \pause
      Var(X) &=& E[X^2] - \left(E[X]\right)^2 = \pause \frac{3}{10} - \left( \frac{1}{2} \right)^2 = 1/20 
    \end{eqnarray*}

    \alert{Complete the algebra at home and check using \texttt{integrate} in R.}
\end{frame}
%%%%%%%%%%%%%%%%%%%%%%%%%%%%%%%%%%%%%%%%%
\begin{frame}[fragile]
  \frametitle{Simulating a Beta$(2, 2)$ Random Variable}
  Our example from above is a special case of the \emph{\alert{Beta distribution}}.
  The command \texttt{rbeta(n, 2, 2)} makes \texttt{n} draws for this RV.
  These simulations agree with our calculations from above:

\begin{knitrout}
\definecolor{shadecolor}{rgb}{0.969, 0.969, 0.969}\color{fgcolor}\begin{kframe}
\begin{alltt}
\hlkwd{set.seed}\hlstd{(}\hlnum{12345}\hlstd{)}
\hlstd{sims} \hlkwb{<-} \hlkwd{rbeta}\hlstd{(}\hlnum{10000}\hlstd{,} \hlnum{2}\hlstd{,} \hlnum{2}\hlstd{)}
\hlkwd{mean}\hlstd{(sims)}
\end{alltt}
\begin{verbatim}
## [1] 0.5007002
\end{verbatim}
\begin{alltt}
\hlkwd{var}\hlstd{(sims)}
\end{alltt}
\begin{verbatim}
## [1] 0.05012776
\end{verbatim}
\end{kframe}
\end{knitrout}


\end{frame}
%%%%%%%%%%%%%%%%%%%%%%%%%%%%%%%%%%%%%%%%%
\begin{frame}[fragile]
  \frametitle{Simulating a Beta$(2, 2)$ Random Variable}
\begin{knitrout}
\definecolor{shadecolor}{rgb}{0.969, 0.969, 0.969}\color{fgcolor}\begin{kframe}
\begin{alltt}
\hlkwd{mean}\hlstd{(sims}\hlopt{^}\hlnum{2}\hlstd{)}
\end{alltt}
\begin{verbatim}
## [1] 0.3008234
\end{verbatim}
\begin{alltt}
\hlkwd{hist}\hlstd{(sims,} \hlkwc{freq} \hlstd{=} \hlnum{FALSE}\hlstd{)}
\end{alltt}
\end{kframe}
\includegraphics[width=\maxwidth]{figure/beta_2_2_more-1} 

\end{knitrout}
\end{frame}
%%%%%%%%%%%%%%%%%%%%%%%%%%%%%%%%%%%%%%%%%
\begin{frame}
  \frametitle{The Uniform Random Variable}
Several of your review questions along with one of your extension questions will involve the so-called \emph{\alert{Uniform Random Variable}}:

\begin{block}{Uniform(0,1) Random Variable}
  $f(x) = 1$ for $x \in [0,1]$, zero otherwise.
\end{block}

\begin{block}{Uniform(a,b) Random Variable}
  $f(x) = 1/(b-a)$ for $x \in [a,b]$, zero otherwise.
\end{block}

\begin{alertblock}{Simulating from a Uniform RV}
  \texttt{runif(n, a, b)} makes \texttt{n} draws from a Uniform$(a,b)$ RV.
\end{alertblock}

\end{frame}
%%%%%%%%%%%%%%%%%%%%%%%%%%%%%%%%%%%%%%%%%
\begin{frame}[fragile]
  \frametitle{Simulating Uniform Random Variables}

  \footnotesize
\begin{knitrout}
\definecolor{shadecolor}{rgb}{0.969, 0.969, 0.969}\color{fgcolor}\begin{kframe}
\begin{alltt}
\hlstd{sims1} \hlkwb{<-} \hlkwd{runif}\hlstd{(}\hlnum{10000}\hlstd{,} \hlnum{0}\hlstd{,} \hlnum{1}\hlstd{)}
\hlstd{sims2} \hlkwb{<-} \hlkwd{runif}\hlstd{(}\hlnum{10000}\hlstd{,} \hlopt{-}\hlnum{1}\hlstd{,} \hlnum{2}\hlstd{)}
\hlkwd{par}\hlstd{(}\hlkwc{mfrow} \hlstd{=} \hlkwd{c}\hlstd{(}\hlnum{1}\hlstd{,} \hlnum{2}\hlstd{))}
\hlkwd{hist}\hlstd{(sims1,} \hlkwc{freq} \hlstd{=} \hlnum{FALSE}\hlstd{)}
\hlkwd{hist}\hlstd{(sims2,} \hlkwc{freq} \hlstd{=} \hlnum{FALSE}\hlstd{)}
\end{alltt}
\end{kframe}
\includegraphics[width=\maxwidth]{figure/sim_unif-1} 
\begin{kframe}\begin{alltt}
\hlkwd{par}\hlstd{(}\hlkwc{mfrow} \hlstd{=} \hlkwd{c}\hlstd{(}\hlnum{1}\hlstd{,}\hlnum{1}\hlstd{))}
\end{alltt}
\end{kframe}
\end{knitrout}
\end{frame}
%%%%%%%%%%%%%%%%%%%%%%%%%%%%%%%%%%%%%%%%%
%\begin{frame}
%\frametitle{Simplest Possible Continuous RV: Uniform$(0,1)$}
%\framesubtitle{You'll look at a generalization, Uniform$(a,b)$ for homework.}
%
%\begin{block}{$X \sim \mbox{Uniform}(0,1)$}
%A Uniform(0,1) RV is equally likely to take on \emph{any value} in the range $[0,1]$ and never takes on a value outside this range.
%\end{block}
%
%\begin{block}{Uniform PDF}
%$f(x) = 1$ for $0\leq x \leq 1$, zero elsewhere.
%\end{block}
%
%
%\end{frame}
%%%%%%%%%%%%%%%%%%%%%%%%%%%%%%%%%%%%%%%%%
%
%
%\begin{frame}
%\frametitle{Uniform$(0,1)$ PDF}
%\centering
%	\includegraphics[scale = 0.6]{./images/uniform_density}
%
%\end{frame}
%
%
%%%%%%%%%%%%%%%%%%%%%%%%%%%%%%%%%%%%%%%%%
%\begin{frame}
%\frametitle{Check that Uniform pdf Integrates to 1}
%
%\centering
%	\includegraphics[scale = 0.4]{./images/uniform_density_shaded}
%\begin{eqnarray*}
%	\int_{-\infty}^{\infty} f(x) \; dx = \int_{0}^1 1 \; dx = \left. x \right|_0^1 = 1 - 0 = 1
%\end{eqnarray*}
%\end{frame}
%
%
%%%%%%%%%%%%%%%%%%%%%%%%%%%%%%%%%%%%%%%%%
%
%
%
%\begin{frame}
%\frametitle{What is the area of the shaded region? \hfill \includegraphics[scale = 0.05]{./images/clicker}}
%\centering
%	\includegraphics[scale = 0.6]{./images/uniform_density_cdf}
%
%\end{frame}
%
%
%%%%%%%%%%%%%%%%%%%%%%%%%%%%%%%%%%%%%%%%%
%
%\begin{frame}
%\frametitle{$F(0.4) = P(X\leq 0.4) = 0.4$}
%\centering
%	\includegraphics[scale = 0.6]{./images/uniform_density_cdf}
%
%\end{frame}
%
%
%%%%%%%%%%%%%%%%%%%%%%%%%%%%%%%%%%%%%%%%%
%\begin{frame}
%\frametitle{Example: Uniform$(0,1)$ RV}
%
%Integrate the pdf, $f(x) = 1$, to get the CDF
%\begin{eqnarray*}
%	F(x_0) =\int_{-\infty}^{x_0} f(x)\; dx = \int_{0}^{x_0} 1\; dx =  \left. x \right|_0^{x_0} =  x_0 - 0 = x_0
%\end{eqnarray*}
%
%\vspace{1em}
%$$ F(x_0) = \left\{ \begin{array}{c} 0, x_0 < 0\\ x_0, 0\leq x_0 \leq 1\\ 1, x_0 > 1   \end{array}\right.$$
%\end{frame}
%
%
%%%%%%%%%%%%%%%%%%%%%%%%%%%%%%%%%%%%%%%%%
%\begin{frame}
%\frametitle{Uniform$(0,1)$ CDF}
%\centering
%	\includegraphics[scale = 0.6]{./images/uniform_CDF}
%
%\end{frame}
%
%
%%%%%%%%%%%%%%%%%%%%%%%%%%%%%%%%%%%%%%%%%
%\begin{frame}
%\frametitle{Example: Uniform$(0,1)$ RV}
%Differentiate the CDF, $F(x_0) = x_0$, to get the pdf
% \begin{eqnarray*}
%	\frac{d}{dx}F(x) =  1 = f(x)
% \end{eqnarray*}
%\end{frame}
%
%
%
%%%%%%%%%%%%%%%%%%%%%%%%%%%%%%%%%%%%%%%%%
%\begin{frame}
%\frametitle{What is $P(0.4 \leq X \leq 0.8)$ if $X\sim \mbox{Uniform}(0,1)$? \hfill \includegraphics[scale = 0.05]{./images/clicker}}
%\centering
%	\includegraphics[scale = 0.6]{./images/uniform_density_interval}
%
%
%\end{frame}
%%%%%%%%%%%%%%%%%%%%%%%%%%%%%%%%%%%%%%%%%
%\begin{frame}
%\frametitle{$F(0.8) = P(X \leq 0.8)$}
%
%\centering
%	\includegraphics[scale = 0.6]{./images/density_interval_cdf1}
%
%\end{frame}
%
%
%%%%%%%%%%%%%%%%%%%%%%%%%%%%%%%%%%%%%%%%%
%\begin{frame}
%\frametitle{$F(0.8) - F(0.4) = $ ?}
%\centering
%	\includegraphics[scale = 0.6]{./images/density_interval_cdf2}
%
%\end{frame}
%
%
%%%%%%%%%%%%%%%%%%%%%%%%%%%%%%%%%%%%%%%%%
%\begin{frame}
%\frametitle{$F(0.8) - F(0.4) = P(0.4 \leq X \leq 0.8) = 0.4$}
%\centering
%	\includegraphics[scale = 0.6]{./images/uniform_density_interval}
%
%
%\end{frame}
%%%%%%%%%%%%%%%%%%%%%%%%%%%%%%%%%%%%%%%%%
%\begin{frame}
%  \frametitle{Example: Uniform(0,1) Random Variable} 
%\begin{eqnarray*}
%	E[X] &=&  \int_{-\infty}^\infty x f(x) \; dx =\pause  \int_{0}^1 x \cdot 1 \; dx \\ \\
%		&=&  \left.\frac{x^2}{2}\right|_0^1 = 1/2  - 0 = 1/2
%\end{eqnarray*}
%\end{frame}
%%%%%%%%%%%%%%%%%%%%%%%%%%%%%%%%%%%%%%%%%
%\begin{frame}
%\frametitle{Expected Value of a Function of a Continuous RV}
%	$$\boxed{E[g(X)] = \int_{-\infty}^\infty g(x) f(x) \; dx}$$
%\end{frame}
%
%
%%%%%%%%%%%%%%%%%%%%%%%%%%%%%%%%%%%%%%%%%
%
%\begin{frame}
%\frametitle{Example: Uniform(0,1) Random Variable \hfill \includegraphics[scale = 0.05]{./images/clicker}}
%	\begin{eqnarray*}
%	E[X^2] &=& \int_{-\infty}^\infty x^2 f(x) \; dx \pause = \int_0^1 x^2 \cdot 1 \; dx\\ \\
%		&=& \left. \frac{x^3}{3}\right|_0^1 = 1/3
%	\end{eqnarray*}
%\end{frame}
%
%
%%%%%%%%%%%%%%%%%%%%%%%%%%%%%%%%%%%%%%%%%
%
%\begin{frame}
%\frametitle{Example: Uniform$(0,1)$ RV}
%\begin{eqnarray*}
% Var(X) &=& E\left[ \left( X - E[X] \right)^2\right] = E[X^2] - \left(E[X]\right)^2\\
% 	&=& 1/3  - (1/2)^2\\ 
% 	&=& 1/12 \\
% 	&\approx& 0.083
%\end{eqnarray*}
%\end{frame}
%
%%%%%%%%%%%%%%%%%%%%%%%%%%%%%%%%%%%%%%%%
%\begin{frame}
%\frametitle{Much More Complicated Without the Shortcut Formula!}
%\begin{eqnarray*}
% Var(X) &=& E\left[ \left( X - E[X] \right)^2\right] = \int_{-\infty}^{\infty} (x - \mu)^2 f(x) \; dx\\ \\
% 	&=&\int_{0}^{1} (x -1/2)^2 \cdot 1 \; dx = \int_{0}^{1} (x^2  - x + 1/4) \; dx \\ \\
% 		&=& \left. \left(\frac{x^3}{3} - \frac{x^2}{2} + \frac{x}{4}  \right)\right|_0^1 = 1/3 - 1/2 + 1/4\\ \\ 
% 			&=& 4/12 - 6/12 + 3/12 = 1/12
%\end{eqnarray*}
%\end{frame}
%
%%%%%%%%%%%%%%%%%%%%%%%%%%%%%%%%%%%%%%%%%
\begin{frame}
  \frametitle{We don't have time to cover these in Econ 103:}

\begin{block}{Joint Density}
$ \displaystyle P(a\leq X \leq b \cap c\leq Y \leq d) = \int_c^d \int_a^b f(x,y) \; dxdy$
\end{block}
\begin{block}{Marginal Densities}
$f_X(x) = \int_{-\infty}^\infty f(x,y)\; dy$, $\;\;\;\;\;\;\; f_Y(y) = \int_{-\infty}^\infty f(x,y)\; dx$
\end{block}
\begin{block}{Independence in Terms of Joint and Marginal Densities}
$f_{XY}(x,y) = f_X(x)f_Y(y)$
\end{block}
\begin{block}{Conditional Density}
$f_{Y|X} = f_{XY}(x,y)/f_X(x)$
\end{block}

\end{frame}

%%%%%%%%%%%%%%%%%%%%%%%%%%%%%%%%%%%%%%%
\begin{frame}
  \frametitle{So where does that leave us?}

  \begin{block}{What We've Accomplished}
    We've covered all the basic properties of RVs on this \href{http://ditraglia.com/Econ103Public/RandomVariablesHandout.pdf}{\textcolor{blue}{\fbox{Handout}}}.
  \end{block}

  \begin{block}{Where are we headed next?}
   Next up is the most important RV of all: the normal RV. 
   After that it's time to do some statistics!
  \end{block}

  \begin{alertblock}{How should you be studying?}
    If you \emph{master} the material on RVs (both continuous and discrete) and in particular the normal RV the rest of the semester will seem easy. 
    If you don't, you're in for a rough time\dots
  \end{alertblock}


\end{frame}
%%%%%%%%%%%%%%%%%%%%%%%%%%%%%%%%%%%%%%%%
