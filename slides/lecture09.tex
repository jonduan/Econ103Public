%!TEX root = lecture_slides.tex
%%%%%%%%%%%%%%%%%%%%%%%%%%%%%%%%%%%%%%%%%
\section{Variance and Standard Deviation of a Random Variable}
%%%%%%%%%%%%%%%%%%%%%%%%%%%%%%%%%%%%%%%%%

\begin{frame}
\frametitle{Variance and Standard Deviation of a RV}
\framesubtitle{The Defs are the same for continuous RVs, but the method of calculating will differ.}

\begin{block}{Variance (Var)}
	$$\sigma^2 = Var(X) = E\left[ (X - \mu)^2\right] = E\left[ (X - E[X])^2\right]$$
\end{block}


\begin{block}{Standard Deviation (SD)}
$$\sigma = \sqrt{\sigma^2} = SD(X)$$
\end{block}


\end{frame}
%%%%%%%%%%%%%%%%%%%%%%%%%%%%%%%%%%%%%%%%
\begin{frame}
\frametitle{Key Point}

\alert{Variance and std.\ dev.\ are \emph{expectations of functions of a RV}} 

\vspace{1em}

It follows that: 
\begin{enumerate}
\item Variance and SD are constants
\item To derive facts about them you can use the facts you know about expected value
\end{enumerate}
\end{frame}
%%%%%%%%%%%%%%%%%%%%%%%%%%%%%%%%%%%%%%%%
\begin{frame}
\frametitle{How To Calculate Variance for Discrete RV?}
\framesubtitle{Remember: it's just a function of $X$!}


Recall that	$\displaystyle \mu = E[X] = \sum_{\mbox{all } x} xp(x)$


\vspace{3em}

$$Var(X) = E\left[ (X - \mu)^2 \right] =\sum_{\mbox{all } x} (x - \mu)^2 p(x)$$


\end{frame}
%%%%%%%%%%%%%%%%%%%%%%%%%%%%%%%%%%%%%%%%
\begin{frame}
\frametitle{Shortcut Formula For Variance}

This is \emph{not} the definition, it's a shortcut for doing calculations:
\begin{eqnarray*}
	Var(X) = E\left[ (X - \mu)^2 \right] = E[X^2] - \left(E[X]\right)^2
\end{eqnarray*}

\alert{We'll prove this in an upcoming lecture.}

\end{frame}

%%%%%%%%%%%%%%%%%%%%%%%%%%%%%%%%%%%%%%%%
\begin{frame}
\frametitle{Example: The Shortcut Formula \hfill \includegraphics[scale = 0.05]{./images/clicker}}
Let $X\sim \mbox{Bernoulli}(1/2)$.
Calculate $Var(X)$.

\pause
\begin{eqnarray*}
    E[X] &=&  0 \times 1/2 + 1 \times 1/2 = 1/2\\ 
    E[X^2] &=&  0^2 \times 1/2 + 1^2 \times 1/2 = 1/2
\end{eqnarray*}
\pause
\[
  \alert{E[X^2] - \left( E[X] \right)^2 = 1/2 - (1/2)^2 =  1/4}
\]
\end{frame}
%%%%%%%%%%%%%%%%%%%%%%%%%%%%%%%%%%%%%%%%


\begin{frame}
\frametitle{Variance of Bernoulli RV -- via the Shortcut Formula}

\begin{block}{Step 1 -- $E[X]$} 
$\mu = E[X] = \displaystyle \sum_{x \in \{0,1\}} p(x) \cdot x = (1-p) \cdot 0 + p \cdot 1 = p$
\end{block}


\begin{block}{Step 2 -- $E[X^2]$} 
\begin{eqnarray*}
	E[X^2] = \sum_{x \in \{0,1\}} x^2 p(x) = 0^2 (1-p) + 1^2 p = p
\end{eqnarray*}
\end{block}


\begin{block}{Step 3 -- Combine with Shortcut Formula} 
\begin{eqnarray*}
	\sigma^2 = Var[X] = E[X^2] - \left(E[X]\right)^2 = \ p - p^2 =  p(1-p)
\end{eqnarray*}
\end{block}


\end{frame}
%%%%%%%%%%%%%%%%%%%%%%%%%%%%%%%%%%%%%%%%
%\begin{frame}
%\frametitle{Variance of Bernoulli RV -- Without Shortcut}
%
%\alert{You will fill in the missing steps on Problem Set 5.}
%
%\begin{eqnarray*}
%	\sigma^2 &=& Var(X) = \sum_{x \in \{0,1\}} (x - \mu)^2 p(x)\\
%	 &=& \sum_{x \in \{0,1\}} (x - p)^2 p(x)\\
%	 &\vdots &\\ 
%	 &=&p(1-p)
%\end{eqnarray*}
%
%\end{frame}
%%%%%%%%%%%%%%%%%%%%%%%%%%%%%%%%%%%%%%%%

%\begin{frame}
%\frametitle{Variance of a Linear Function  \hfill \includegraphics[scale = 0.05]{./images/clicker} }
%Suppose $X$ is a random variable with $Var(X) = \sigma^2$ and $a,b$ are constants. What is $Var(a + bX)$? 
%\begin{enumerate}[(a)]
%	\item $\sigma^2$
%	\item $a + \sigma^2$
%	\item $b \sigma^2$
%	\item $a + b \sigma^2$
%	\item $b^2 \sigma^2$
%\end{enumerate}
%
%\end{frame}
%%%%%%%%%%%%%%%%%%%%%%%%%%%%%%%%%%%%%%%%
\begin{frame}
\frametitle{Variance of a Linear Transformation}

\begin{eqnarray*}
 Var(a + bX) &=& E\left[\left\{(a+bX) - E(a+bX)\right\}^2 \right] \\ 
 	&=& E\left[\left\{(a+bX) - (a+bE[X])\right\}^2 \right] \\
 	&=&E\left[\left(bX - bE[X]\right)^2 \right] \\ 
 	&=&E[b^2 (X - E[X])^2]\\ 
 	&=& b^2 E[(X-E[X])^2]\\ 
 	&=& b^2 Var(X) = b^2 \sigma^2
\end{eqnarray*}
\alert{The key point here is that variance is defined in terms of expectation and expectation is linear.}

\end{frame}
%%%%%%%%%%%%%%%%%%%%%%%%%%%%%%%%%%%%%%%%
\begin{frame}
	\frametitle{Variance and SD are \emph{NOT} Linear}

\begin{eqnarray*}
Var(a + bX) &= &b^2 \sigma^2 \\\\
	SD(a + bX)&=& |b| \sigma
\end{eqnarray*}

\vspace{2em}
\begin{block}{These should look familiar from the related results for sample variance and std.\ dev. that you worked out on an earlier problem set.}

\end{block}

\end{frame}
%%%%%%%%%%%%%%%%%%%%%%%%%%%%%%%%%%%%%%%%
\section{Binomial Random Variable}
%%%%%%%%%%%%%%%%%%%%%%%%%%%%%%%%%%%%%%%%

\begin{frame}
\frametitle{Binomial Random Variable}
Let $X = $ the sum of $n$ independent Bernoulli trials, each with probability of success $p$.  \alert{Then we say that:
		$X \sim \mbox{Binomial}(n,p)$} 

\vspace{2em}
\begin{block}{Parameters}
$p =$ probability of ``success,'' $n=$ \# of trials
\end{block}
\begin{block}{Support} 
$\{0, 1, 2, \hdots, n\}$ 
\end{block}
\begin{block}{Probability Mass Function (pmf)} 
$$p(x) = {n \choose x} p^x (1-p)^{n-x}$$ 
\end{block}
\end{frame}
%%%%%%%%%%%%%%%%%%%%%%%%%%%%%%%%%%%%%%%%

\begin{frame}
	\frametitle{\href{https://fditraglia.shinyapps.io/binom_cdf/}{http://fditraglia.shinyapps.io/binom\_cdf/}}
\framesubtitle{Try playing around with all three sliders. If you set the second to 1 you get a Bernoulli.}

\begin{figure}
	\fbox{\includegraphics[scale = 0.2]{./images/binom_cdf_screenshot2}}
\end{figure}

\end{frame}

%%%%%%%%%%%%%%%%%%%%%%%%%%%%%%%%%%%%%%%%


%\begin{frame}
%	\frametitle{\href{http://fditraglia.github.com/Econ103Public/Rtutorials/Bernoulli_Binomial.html}{\small http://fditraglia.github.com/Econ103Public/Rtutorials/Bernoulli\_Binomial.html}}
%
%\framesubtitle{Source Code on my \href{https://github.com/fditraglia/Econ103Public/blob/master/Rtutorials/Bernoulli_Binomial.R}{\fbox{Github Page}}}
%
%
%
%\begin{figure}
%	\fbox{\includegraphics[scale = 0.15]{./images/binom_bernoulli_sim_screenshot}}
%\end{figure}
%
%\begin{alertblock}{Don't forget this!}
% Binomial RV counts up the \emph{total} number of successes (ones) in $n$ indep.\ Bernoulli trials, each with prob.\ of success $p$.
%\end{alertblock}
%
%\end{frame}

%%%%%%%%%%%%%%%%%%%%%%%%%%%%%%%%%%%%%%%%

\begin{frame}
\frametitle{Where does the Binomial pmf come from? \includegraphics[scale = 0.05]{./images/clicker} }
\begin{block}{Question}
Suppose we flip a fair coin 3 times. What is the probability that we get exactly 2 heads?
\end{block}

\pause

\begin{block}{Answer}
Three basic outcomes make up this event: $\{HHT, HTH, THH\}$, each has probability $1/8 = 1/2 \times 1/2 \times 1/2$. Basic outcomes are mutually exclusive, so sum to get \alert{$3/8 = 0.375$}
\end{block}

\end{frame}
%%%%%%%%%%%%%%%%%%%%%%%%%%%%%%%%%%%%%%%%
%%%%%%%%%%%%%%%%%%%%%%%%%%%%%%%%%%%%%%%%
\begin{frame}
\frametitle{Where does the Binomial pmf come from?}
\begin{block}{Question}
Suppose we flip an \emph{unfair} coin 3 times, where the probability of heads is 1/3. What is the probability that we get exactly 2 heads?
\end{block}



\begin{block}{Answer}
  No longer true that \emph{all} basic outcomes are equally likely, but those with exactly two heads \emph{\alert{still are}} 
	\begin{eqnarray*}
	 P(HHT) &=&  (1/3)^2 (1 - 1/3) =  2/27\\ 
	 P(THH) &=&2/27\\ 
	 P(HTH) &=&2/27 
	\end{eqnarray*}
	Summing gives \alert{$2/9 \approx 0.22$}
\end{block}
\end{frame}
%%%%%%%%%%%%%%%%%%%%%%%%%%%%%%%%%%%%%%%%
\begin{frame}
  \frametitle{Where does the Binomial pmf come from?}
\framesubtitle{Starting to see a pattern?}

Suppose we flip an unfair coin \emph{4} times, where the probability of heads is 1/3. What is the probability that we get exactly 2 heads?

\vspace{2em}



\begin{columns}
\column{0.35\textwidth}
\begin{tabular}{cc}
HHTT&TTHH\\
HTHT&THTH\\
HTTH&THHT
\end{tabular}

\column{0.55\textwidth}
\alert{Six equally likely, mutually exclusive basic outcomes make up this event:}
$${4\choose 2}(1/3)^2(2/3)^2$$
\end{columns}

\end{frame}
%%%%%%%%%%%%%%%%%%%%%%%%%%%%%%%%%%%%%%%%%%%%%%%%%%%%%%%%%
\begin{frame}
  \frametitle{R Commands for Binomial$(n,p)$ RV}

  \begin{block}{Probability Mass Function}
    \alert{\texttt{dbinom(x, size, prob)}}, where \alert{\texttt{size}} is $n$ and \alert{\texttt{prob}} is $p$
  \end{block}

  \begin{block}{Cumulative Distribution Function}
    \alert{\texttt{pbinom(q, size, prob)}}, where \alert{\texttt{q}} is $x_0$, \alert{\texttt{size}} is $n$ and \alert{\texttt{prob}} is $p$
  \end{block}

  \begin{block}{Make Random Draws}
    \alert{\texttt{rbinom(n, size, prob)}}, where \alert{\texttt{n}} is the number of draws, \alert{\texttt{size}} is $n$ and \alert{\texttt{prob}} is $p$
  \end{block}

\end{frame}
%%%%%%%%%%%%%%%%%%%%%%%%%%%%%%%%%%%%%%%%
\begin{frame}[fragile]
  \footnotesize

\begin{knitrout}
\definecolor{shadecolor}{rgb}{0.969, 0.969, 0.969}\color{fgcolor}\begin{kframe}
\begin{alltt}
\hlstd{x} \hlkwb{<-} \hlnum{0}\hlopt{:}\hlnum{10}
\hlstd{px} \hlkwb{<-} \hlkwd{dbinom}\hlstd{(x,} \hlkwc{size} \hlstd{=} \hlnum{10}\hlstd{,} \hlkwc{prob} \hlstd{=} \hlnum{0.3}\hlstd{)}
\hlstd{x0} \hlkwb{<-} \hlkwd{seq}\hlstd{(}\hlkwc{from} \hlstd{=} \hlopt{-}\hlnum{2}\hlstd{,} \hlkwc{to} \hlstd{=} \hlnum{12}\hlstd{,} \hlkwc{by} \hlstd{=} \hlnum{0.01}\hlstd{)}
\hlstd{Fx} \hlkwb{<-} \hlkwd{pbinom}\hlstd{(x0,} \hlkwc{size} \hlstd{=} \hlnum{10}\hlstd{,} \hlkwc{prob} \hlstd{=} \hlnum{0.3}\hlstd{)}
\hlkwd{par}\hlstd{(}\hlkwc{mfrow} \hlstd{=} \hlkwd{c}\hlstd{(}\hlnum{1}\hlstd{,} \hlnum{2}\hlstd{))}
\hlkwd{plot}\hlstd{(x, px,} \hlkwc{type} \hlstd{=} \hlstr{'h'}\hlstd{,} \hlkwc{ylab} \hlstd{=} \hlstr{'p(x)'}\hlstd{,} \hlkwc{main} \hlstd{=} \hlstr{'Binom(10, 0.3) pmf'}\hlstd{)}
\hlkwd{plot}\hlstd{(x0, Fx,} \hlkwc{type} \hlstd{=} \hlstr{'l'}\hlstd{,} \hlkwc{ylab} \hlstd{=} \hlstr{'F(x)'}\hlstd{,} \hlkwc{main} \hlstd{=} \hlstr{'Binom(10, 0.3) CDF'}\hlstd{)}
\end{alltt}
\end{kframe}
\includegraphics[width=\maxwidth]{figure/binom_pmf_cdf-1} 
\begin{kframe}\begin{alltt}
\hlkwd{par}\hlstd{(}\hlkwc{mfrow} \hlstd{=} \hlkwd{c}\hlstd{(}\hlnum{1}\hlstd{,} \hlnum{1}\hlstd{))}
\end{alltt}
\end{kframe}
\end{knitrout}

\end{frame}
%%%%%%%%%%%%%%%%%%%%%%%%%%%%%%%%%%%%%%%%
\begin{frame}[fragile]
  \footnotesize
\begin{knitrout}
\definecolor{shadecolor}{rgb}{0.969, 0.969, 0.969}\color{fgcolor}\begin{kframe}
\begin{alltt}
\hlkwd{set.seed}\hlstd{(}\hlnum{5545}\hlstd{)}
\hlstd{sims} \hlkwb{<-} \hlkwd{rbinom}\hlstd{(}\hlnum{100}\hlstd{,} \hlkwc{size} \hlstd{=} \hlnum{10}\hlstd{,} \hlkwc{prob} \hlstd{=} \hlnum{0.3}\hlstd{)}
\hlkwd{par}\hlstd{(}\hlkwc{mfrow} \hlstd{=} \hlkwd{c}\hlstd{(}\hlnum{1}\hlstd{,} \hlnum{2}\hlstd{))}
\hlstd{rel_freq} \hlkwb{<-} \hlkwd{prop.table}\hlstd{(}\hlkwd{table}\hlstd{(sims))}
\hlkwd{plot}\hlstd{(rel_freq,} \hlkwc{main} \hlstd{=} \hlstr{'100 Binom(10, 0.3) sims'}\hlstd{,}
     \hlkwc{ylab} \hlstd{=} \hlstr{'Relative Frequency'}\hlstd{)}
\hlkwd{plot}\hlstd{(x, px,} \hlkwc{type} \hlstd{=} \hlstr{'h'}\hlstd{,} \hlkwc{ylab} \hlstd{=} \hlstr{'p(x)'}\hlstd{,} \hlkwc{main} \hlstd{=} \hlstr{'Binomial(10, 0.3) pmf'}\hlstd{)}
\end{alltt}
\end{kframe}
\includegraphics[width=\maxwidth]{figure/binom_sims_pmf-1} 
\begin{kframe}\begin{alltt}
\hlkwd{par}\hlstd{(}\hlkwc{mfrow} \hlstd{=} \hlkwd{c}\hlstd{(}\hlnum{1}\hlstd{,} \hlnum{1}\hlstd{))}
\end{alltt}
\end{kframe}
\end{knitrout}
\end{frame}
%%%%%%%%%%%%%%%%%%%%%%%%%%%%%%%%%%%%%%%%
